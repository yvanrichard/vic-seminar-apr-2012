%% This is file `elsarticle-template-2-harv.tex',
%%
%% Copyright 2009 Elsevier Ltd
%%
%% This file is part of the 'Elsarticle Bundle'.
%% ---------------------------------------------
%%
%% It may be distributed under the conditions of the LaTeX Project Public
%% License, either version 1.2 of this license or (at your option) any
%% later version.  The latest version of this license is in
%%    http://www.latex-project.org/lppl.txt
%% and version 1.2 or later is part of all distributions of LaTeX
%% version 1999/12/01 or later.
%%
%% The list of all files belonging to the 'Elsarticle Bundle' is
%% given in the file `manifest.txt'.
%%
%% Template article for Elsevier's document class `elsarticle'
%% with harvard style bibliographic references
%%
%% $Id: elsarticle-template-2-harv.tex 155 2009-10-08 05:35:05Z rishi $
%% $URL: http://lenova.river-valley.com/svn/elsbst/trunk/elsarticle-template-2-harv.tex $
%%
\documentclass[preprint,authoryear,12pt]{elsarticle}

%% Use the option review to obtain double line spacing
%% \documentclass[authoryear,preprint,review,12pt]{elsarticle}

%% Use the options 1p,twocolumn; 3p; 3p,twocolumn; 5p; or 5p,twocolumn
%% for a journal layout:
%% \documentclass[final,authoryear,1p,times]{elsarticle}
%% \documentclass[final,authoryear,1p,times,twocolumn]{elsarticle}
%% \documentclass[final,authoryear,3p,times]{elsarticle}
%% \documentclass[final,authoryear,3p,times,twocolumn]{elsarticle}
%% \documentclass[final,authoryear,5p,times]{elsarticle}
%% \documentclass[final,authoryear,5p,times,twocolumn]{elsarticle}

%% if you use PostScript figures in your article
%% use the graphics package for simple commands
%% \usepackage{graphics}
%% or use the graphicx package for more complicated commands
%% \usepackage{graphicx}
%% or use the epsfig package if you prefer to use the old commands
%% \usepackage{epsfig}

%% The amssymb package provides various useful mathematical symbols
\usepackage{amssymb}
%% The amsthm package provides extended theorem environments
%% \usepackage{amsthm}

%% The lineno packages adds line numbers. Start line numbering with
%% \begin{linenumbers}, end it with \end{linenumbers}. Or switch it on
%% for the whole article with \linenumbers after \end{frontmatter}.
\usepackage{lineno}
\usepackage{amsmath}
%% natbib.sty is loaded by default. However, natbib options can be
%% provided with \biboptions{...} command. Following options are
%% valid:

%%   round  -  round parentheses are used (default)
%%   square -  square brackets are used   [option]
%%   curly  -  curly braces are used      {option}
%%   angle  -  angle brackets are used    <option>
%%   semicolon  -  multiple citations separated by semi-colon (default)
%%   colon  - same as semicolon, an earlier confusion
%%   comma  -  separated by comma
%%   authoryear - selects author-year citations (default)
%%   numbers-  selects numerical citations
%%   super  -  numerical citations as superscripts
%%   sort   -  sorts multiple citations according to order in ref. list
%%   sort&compress   -  like sort, but also compresses numerical citations
%%   compress - compresses without sorting
%%   longnamesfirst  -  makes first citation full author list
%%
%% \biboptions{longnamesfirst,comma}

% \biboptions{}

\journal{Ecological Modelling}

\begin{document}

\begin{frontmatter}

%% Title, authors and addresses

%% use the tnoteref command within \title for footnotes;
%% use the tnotetext command for the associated footnote;
%% use the fnref command within \author or \address for footnotes;
%% use the fntext command for the associated footnote;
%% use the corref command within \author for corresponding author footnotes;
%% use the cortext command for the associated footnote;
%% use the ead command for the email address,
%% and the form \ead[url] for the home page:
%%
%% \title{Title\tnoteref{label1}}
%% \tnotetext[label1]{}
%% \author{Name\corref{cor1}\fnref{label2}}
%% \ead{email address}
%% \ead[url]{home page}
%% \fntext[label2]{}
%% \cortext[cor1]{}
%% \address{Address\fnref{label3}}
%% \fntext[label3]{}

\title{The arithmetic mean is not appropriate in multiplicative processes}

%% use optional labels to link authors explicitly to addresses:
%% \author[label1,label2]{<author name>}
%% \address[label1]{<address>}
%% \address[label2]{<address>}

\author{Yvan ~Richard\corref{cor1}}
\ead{yvan@dragonfly.co.nz}
\ead[url]{http://www.dragonfly.co.nz}
\cortext[cor1]{Corresponding author}
\author{Edward R. Abraham}
\author{Finlay N. Thompson}
\address{Dragonfly Science, PO Box 27535, Wellington 6141, New Zealand}

\begin{abstract}
%% Text of abstract
  Most processes studied in ecological modelling are fundamentally
  multiplicative rather than additive. The average of rates such as
  population growth and survival should therefore be represented by
  their geometric mean. However, the statistical distributions that
  are most commonly used are parametrised according to their
  arithmetic mean. We argue that this mean can be easily
  misinterpreted, introduce biases in many models, and lead to wrong
  conclusions being hastily reached.

  We focus in this short note on two examples. First, the widespread
  belief that increased stochasticity necessarily decreases population
  growth is due to the fact that the arithmetic mean is kept constant
  in models in which temporal variability is varied. This choice of
  mean does not rely on sound biological basis, and this folklore is
  simply a statistical artefact. We propose a distribution that allows
  the level of environmental stochasticity to be varied in models
  without changing the population growth rate. Our other focus is that
  for demographic rates such as survival, the reported average is
  almost always the arithmetic mean, and using this mean in population
  models results in positive bias in long-term population growth.

.  how mean survival
  rates are generally calculated.is convenient as a
  measure of central tendency, the geometric mean is the actual driver
  of multiplicative processes It is widely recognised that
  environmental stochasticity leads to a decrease in the long-term
  population growth rate. We argue that this effect is a modelling
  artefact, due to the way stochasticity is integrated in models of
  population dynamics, and more generally because the arithmetic mean
  is the type of average. We propose two distributions that preserve
  the geometric mean of demographic rates regardless of the level of
  environmental stochasticity. There is no sound basis to fix the
  arithmetic mean and change the level, and that this leads to a
  negative bias in projected growth rates. The fundamental problem
  lies in the fact that the geometric mean of the demographic
  parameters is lowered with increased stochasticity when using the
  distributions commonly used in demographic models. However, this
  effect does not have a real ecological basis. The geometric mean can
  be kept constant regardless the level of stochasticity if the
  geometric mean is raised to the power of a random variable whose
  distribution has an arithmetic mean of 1 and is positive. A gamma
  distribution with equal shape and rate parameters satisfies this
  condition, and the value of these parameters can be easily
  determined to match the desired level of stochasticity.  We
  illustrate our point by running simulations based on published
  annual adult survival rates calculated from 43 years of ringing and
  dead recovery of European shag (\textit{Phalacrocorax aristotelis}).
\end{abstract}

\begin{keyword}
%% keywords here, in the form: keyword \sep keyword
  average \sep arithmetic mean \sep geometric mean \sep stochasticity
  \sep population modelling \sep stochastic growth rate \sep survival
  \sep bias
%% MSC codes here, in the form: \MSC code \sep code
%% or \MSC[2008] code \sep code (2000 is the default)
\end{keyword}

\end{frontmatter}

\linenumbers

%% main text
\section{Introduction}
\label{sec:intro}
The rule of inequality of arithmetic and geometric means, also called
the AM-GM inequality, states that the geometric mean of a set of
non-negative real numbers is less than the arithmetic mean, with the
two means being equal only when all the numbers are the same. 
The difference increases with variance, as for any distribution, 
the geometric ($G$) and the arithmetic ($A$)
means are related by the following formula
\citep{lewontin_population_1969}:

\[ G \cong A \exp \left(\frac{-\sigma^2}{2 A^2} \right) \]

The capacity of each mean to be a useful measure of central tendency
depends on the nature of the process being examined. In additive
processes, the arithmetic mean is most useful, whereas in
multiplicative processes, the geometric mean is best.

In ecology, most processes are multiplicative. For instance, with a
constant growth rate $\lambda$ and in absence of density dependence,
the population size after $T$ years is
\[ N_{T} = N_{0} \lambda^T \]
Similarly the number of survivors $S$ under a constant annual survival
rate $\phi$ is \[ S_{T} = S_{0} \phi^T \]

However, environments in which individuals are born and die always
vary over time, resulting in variations in vital rates. The
distribution of a variable rate can be summarised by an average and a
spread. In order to be useful, the average should be reported so that
the process that was modelled can be re-created and simulated. In the
case of survival, the average should be the geometric mean $G$, so
that
\[ N_{T} = N_{0} \prod_{t=1}^{T} \phi_{t} = N_{0} G^T \]
Unfortunately, common methods for estimating survival, such as program
MARK for capture-recapture data \citep{white_program_1999}, report the
arithmetic mean. Using this mean naively in a deterministic matrix
population model for example may therefore lead to the overall
survival of individuals to be overestimated. Although the bias may not
be large when the temporal variation in the original data is small,
the multiplicative nature of survival processes is likely to be
significant when population projections are made on a large time
frame.


In stochastic models, the correct geometric mean can be obtained from
the distribution of survival rates defined from both the estimated
arithmetic mean and process variance, provided these two parameters
are correctly estimated. However, one might want to simulate the
dynamics of the population under increased stochasticity, either
because the data have been collected over few years and the variation
over time is suspected to be non-representative, or because of
suspected future increase due to climate change for example.

Many studies have modelled the demographic impact of increased
stochasticity by changing the variance of demographic rates while
keeping the arithmetic mean fixed
\citep[e.g.,][]{frederiksen_demographic_2008,schmutz_stochastic_2009,samaranayaka_modelling_2010}.
Because increased stochasticity decreases the geometric mean
relatively to the arithmetic mean, it is normal that these studies
found lower stochastic growth rates, and some authors have concluded
from this result that climate change will globally decrease
populations' growth rate. It is surprising that the focus has been on
maintaining the arithmetic mean fixed as it has little value for rates
and multiplicative processes. The arithmetic mean of physical
processes such as rainfall and temperature may well be constant,
species are likely to respond non-linearly to changes from the mean
\citep{drake_population_2005}, and the decision of keeping the
arithmetic mean constant instead of the geometric one is not based on
biological ground.

the stochastic population
growth rate ($\lambda_{s}$) may be found via simulation using the
following formula \citep{caswell_matrix_2001}

\[
\lambda_{s} = \exp \Big( \frac{1}{T} \big( \ln(N_{T}) -
  \ln(N_{0}) \big) \Big)
\]

with $N_{i}$ being the population size at time $i$, and $T$ the number
of time steps in the model. This formula is simply the geometric mean
of the population growth rate at each time step ($\lambda_{t \to t+1}
= N_{t+1}/N_{t}$). Similarly, the number of survivors after $T$ time
steps is $N_{T} = N_{0} S_{1} S_{2} ... S_{T} = N_{0} G^T$ with $G$
being the geometric mean of $S$. The deterministic growth rate should
therefore be calculated using the geometric mean of $S$, not the
arithmetic mean as commonly done.




Population growth rates always vary in time, either due to the
realisation of discrete events such births and deaths (termed
demographic stochasticity), or due to the variations in the mean
demographic parameters following variations in the environment
(environmental stochasticity). Except when population sizes are very
small, the latter is generally the dominant source of
variability. Stochasticity is important to take into account in models
of population dynamics as populations can decline by chance over a
period of time and eventually go extinct even when their intrinsic
growth rate is positive, and vice-versa. However, modelling
environmental stochasticity is not trivial. This is mainly due to the
fact that the statistical distribution of demographic parameters
varying in time and the degree of temporal autocorrelation of these
variations are typically unknown, because most time series of
demographic estimates are only a few years long.

In the case of survival rates, modelling the trajectory of populations
over time often implies drawing each simulated year the value of
survival rates from a distribution of mean equal to the geometric mean
of survival rates estimated from consecutive years. Various
distributions have been used. For instance, the normal \citep[e.g.,
][]{winemiller_life_2002} and the log-normal distributions are the two
options proposed in the statistical package RAMAS-GIS
\citep{akcakaya_ramas_2002} often used for Population Viability
Analysis. The beta, the logit-normal, the probit-normal, the
lognormal-power, and the truncated-gamma distributions are also
popular \citep{kaye_effect_2003,samaranayaka_modelling_2010}. However,
the long-term stochastic population growth rate obtained by simulation
is robust to the choice of these distributions
\citep{samaranayaka_modelling_2010}.

In a classic scenario, one wants to estimate the growth rate of a
population in order to assess if this population is at risk from
extinction or how its dynamics would respond to various
managements. Data on demographic parameters are collected in the field
for a few years, and then analysed, providing for each parameter a
mean estimate and its associated standard error. These estimates are
then plugged in a population model, often a matrix population model
\citep{caswell_matrix_2001}, and many trajectories of the population
are simulated over a certain time frame in order to calculate the
growth rate. Environmental stochasticity is included in the model by
drawing each year the value of the demographic parameters from one of
the previously described distributions, parametrised to match the mean
and the process variance (removing the sampling variance) estimated
from field data. Because field data are typically collected over a
small number of years, the temporal variation in the demographic
parameters is often imprecisely known and modellers might run the
simulations with various levels of stochasticity. By doing so, many
modellers realised that increasing stochasticity leads to a decrease
of the stochastic growth rate, and it has been acknowledged that this
decrease is caused by a decrease in the geometric mean of the
distribution of the parameters, which is driving the dynamics.

This fact is now widely accepted, and its evolutionary consequences
have been explored, such as the negative correlation between the
temporal variability of a parameter and the sensitivity of growth rate
to this parameter \citep{pfister_patterns_1998}. 

If estimates of vital rates are only available for few years, as it is
commonly the case, their probability distribution cannot be determined
from the data, and modellers need to choose one distribution, assumed
to approximate reality. Various distributions are commonly used. For
instance, the normal \citep[e.g.,][]{winemiller_life_2002} and the
log-normal distributions are the two possibilities proposed in the
software package RAMAS-GIS \citep{akcakaya_ramas_2002} often used
for Population Viability Analysis. The beta, the logit-normal, the
probit-normal, the lognormal-power, and the truncated-gamma
distributions are also popular
\citep{kaye_effect_2003,samaranayaka_modelling_2010}. For demographic
rates, it is more
convenient to choose a distribution bound between 0 and 1. For
long-lived species, i.e. with high annual survival rate, normal or
lognormal distributions for instance can provide rates higher than 1,
and truncation is not preferable, as it can introduce biases if not
corrected \citep{kaye_effect_2003}. It has been
however shown that the long-term stochastic population growth rate
obtained by simulation is robust to the choice between the beta,
the logit-normal, the probit-normal, and the lognormal-power distributions
\citep{samaranayaka_modelling_2010}. But is this growth rate correct?

If estimates of survival rates are available for $n$ years ($S_1$,
$S_2$, ..., $S_n$), then the overall annual survival rate is the
geometric mean:
\[ G = \sqrt[n]{\prod_{i=1}^{n} S_{i}} = \exp(\frac{1}{n}\sum_{i=1}^{n}\ln(S_{i})) \]

The choice of the distribution of annual survival rates should be
therefore made so that simulated individuals have the same chance of
survival as in the observed population. Statistically speaking, the
geometric mean of the sample drawn from this distribution should
therefore match that from the observed population. However, it is
interesting that all commonly used distributions compared in
\citep{samaranayaka_modelling_2010} lead to lower geometric means
because the distributions are parametrised according to the arithmetic
mean. It is well known that the geometric is always less than the
arithmetic one, and that the difference We argue that this is a
statistical artifact, not a fact of life. Indeed, the geometric mean
can be kept constant regardless of the level of stochasticity. One
solution we identified is to draw samples of survival rates following
\[ S = \hat{G}^x \]
with $\hat{G}$ being the geometric mean of observed survival rates,
and $x$ a random variable distributed from a positive distribution
having an arithmetic mean equal to 1. One suitable distribution that
can satisfy these properties is the gamma distribution, characterised
by the following probability density function (pdf)
\[P(x) = x^{k-1}\rho^{k}\frac{e^{-x \rho}}{\Gamma(k)} \] 
with $\Gamma$ being the gamma function, $k$ the shape, and $\rho$ the
rate (note that the pdf function is sometimes parametrised with the
scale, being the inverse the rate). The arithmetic mean of the gamma
distribution is $k/\rho$ and can therefore be constrained to 1
if the rate $\rho$ is set equal to the shape $k$. The variance of $S$
can therefore be controlled with only one parameter, the shape of the
gamma distribution $k$ (Figure \ref{fig:gamma}). 


\begin{figure}[htbp]
\begin{center}
\includegraphics[width=.55\textwidth]{plots/gamma_shapes.pdf}
\end{center}
\caption{Gamma-power distribution of simulated annual survival rate under various
  levels of environmental stochasticity. The geometric mean is fixed
  here to 0.7 and the shape of the gamma distribution is equal to 1, 2, 5, and 10. }
\label{fig:gamma}
\end{figure}


To illustrate the problem, we used a published time series of 40 annual
adult survival rates of European shag \citep[\textit{Phalacrocorax
  aristotelis};][]{frederiksen_demographic_2008}. These estimates,
obtained from manually digitising the Figure 5 in
\citet{frederiksen_demographic_2008} with the software g3data
(http://github.com/pn2200/g3data), were estimated from an unconstrained
time-varying model and can therefore be considered as `observed'
values.

We used the same simple population model as in
\citet{frederiksen_demographic_2008}, with 3 age classes (1 year-old,
2 year-old, and adults), with a pre-breeding census and only for
females.  The transition matrix is
\[
M = \begin{bmatrix}
0    & 0    & f   \\ 
S_{1} & 0    & 0   \\ 
0    & S_{2} & S_{a}
\end{bmatrix}
\]
with $S_{1,2,a}$ being the annual survival rate of 1 year-old, 2
year-old, and adults (from 3 year-old) respectively, $f$ the number of
chicks per female surviving to age 1.

For illustration only, we assumed that $f$, $S_{1}$, and $S_{2}$ were
constant over time, and were chosen so that the simulated population
trajectories would be approximately constant. The chosen values were
$f=0.9$, $S_{1}=0.5$, and $S_{2}=0.7$. By using these values and by
fixing $S_{a}$ to the geometric mean of the observed values (0.832),
the deterministic long-term growth rate of the population without
carrying capacity is $\lambda = 1.064$.

%%  3 figures to show fit of models to 'observed' survival rates

%% 2 figures; 1 to show simulation trajectories and 1 for distribution
%% of Ns after 50 years under the 3 models (bootstrap, gamma-power, logit-normal)

- ok if mean survival is the arithmetic mean (as provided by MARK)





\section{Material and methods}
\label{sec:methods}

\section{Results}
\label{sec:results}

\section{Discussion}
\label{sec:discussion}

\section{Conclusions}
\label{sec:conclusions}

%% The Appendices part is started with the command \appendix;
%% appendix sections are then done as normal sections
%% \appendix

\section{Parametrisation of the gamma distribution}
%% \label{}

%% References
%%
%% Following citation commands can be used in the body text:
%%
%%  \citet{key}  ==>>  Jones et al. (1990)
%%  \citep{key}  ==>>  (Jones et al., 1990)
%%
%% Multiple citations as normal:
%% \citep{key1,key2}         ==>> (Jones et al., 1990; Smith, 1989)
%%                            or  (Jones et al., 1990, 1991)
%%                            or  (Jones et al., 1990a,b)
%% \cite{key} is the equivalent of \citet{key} in author-year mode
%%
%% Full author lists may be forced with \citet* or \citep*, e.g.
%%   \citep*{key}            ==>> (Jones, Baker, and Williams, 1990)
%%
%% Optional notes as:
%%   \citep[chap. 2]{key}    ==>> (Jones et al., 1990, chap. 2)
%%   \citep[e.g.,][]{key}    ==>> (e.g., Jones et al., 1990)
%%   \citep[see][pg. 34]{key}==>> (see Jones et al., 1990, pg. 34)
%%  (Note: in standard LaTeX, only one note is allowed, after the ref.
%%   Here, one note is like the standard, two make pre- and post-notes.)
%%
%%   \citealt{key}          ==>> Jones et al. 1990
%%   \citealt*{key}         ==>> Jones, Baker, and Williams 1990
%%   \citealp{key}          ==>> Jones et al., 1990
%%   \citealp*{key}         ==>> Jones, Baker, and Williams, 1990
%%
%% Additional citation possibilities
%%   \citeauthor{key}       ==>> Jones et al.
%%   \citeauthor*{key}      ==>> Jones, Baker, and Williams
%%   \citeyear{key}         ==>> 1990
%%   \citeyearpar{key}      ==>> (1990)
%%   \citetext{priv. comm.} ==>> (priv. comm.)
%%   \citenum{key}          ==>> 11 [non-superscripted]
%% Note: full author lists depends on whether the bib style supports them;
%%       if not, the abbreviated list is printed even when full requested.
%%
%% For names like della Robbia at the start of a sentence, use
%%   \Citet{dRob98}         ==>> Della Robbia (1998)
%%   \Citep{dRob98}         ==>> (Della Robbia, 1998)
%%   \Citeauthor{dRob98}    ==>> Della Robbia


%% References with bibTeX database:

\bibliographystyle{model2-names}
\bibliography{bib}

%% Authors are advised to submit their bibtex database files. They are
%% requested to list a bibtex style file in the manuscript if they do
%% not want to use model2-names.bst.

%% References without bibTeX database:

% \begin{thebibliography}{00}

%% \bibitem must have one of the following forms:
%%   \bibitem[Jones et al.(1990)]{key}...
%%   \bibitem[Jones et al.(1990)Jones, Baker, and Williams]{key}...
%%   \bibitem[Jones et al., 1990]{key}...
%%   \bibitem[\protect\citeauthoryear{Jones, Baker, and Williams}{Jones
%%       et al.}{1990}]{key}...
%%   \bibitem[\protect\citeauthoryear{Jones et al.}{1990}]{key}...
%%   \bibitem[\protect\astroncite{Jones et al.}{1990}]{key}...
%%   \bibitem[\protect\citename{Jones et al., }1990]{key}...
%%   \harvarditem[Jones et al.]{Jones, Baker, and Williams}{1990}{key}...
%%

% \bibitem[ ()]{}

% \end{thebibliography}

\end{document}

%%
%% End of file `elsarticle-template-2-harv.tex'.
